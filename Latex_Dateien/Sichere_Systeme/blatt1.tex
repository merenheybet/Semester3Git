\documentclass[12pt]{article}
\usepackage{graphicx} % Required for inserting images
\usepackage{indentfirst}
\usepackage[a4paper, margin=3cm]{geometry}
\usepackage[main=ngerman]{babel}
\usepackage{hyperref}
\usepackage[shortlabels]{enumitem}

\setlength{\parskip}{5pt}
\graphicspath{./images/}
\hypersetup{linktoc=all}

\title{Sichere Systeme \\
Aufgabe 1 - Gruppe 1}
\author{Metin Eren Heybet - lu08mika}
\date{24 Oktober 2024}

\begin{document}

    \maketitle

    \section{Schutzziele}

    \begin{enumerate}[(a)]
        \item Die Verfügbarkeit von dem Netzwerk des Nachbarn wird hier wegen (dem?) Jamming verletzt.
        \item Hier werden die Vertraulichkeit, Integrität des Zustandes und Verfügbarkeit verletzt.
        \begin{itemize}
            \item Die Vertraulichkeit: Die personenbezogene Daten sind von Angreifer erreicht geworden und wurden eine Kopie davon erstellt 
            \item Die Integrität des Zustandes: Die Originaleinträge wurden aus der Datenbank gelöscht.
            \item Verfügbarkeit: Die Informationen sind zurzeit nicht erreichbar.
        
            \item Außerdem könnte argumentiert werden, dass die Anonymität der Kunden verletzt wurden.
        \end{itemize}

        \item Vertraulichkeit und möglicherweise die Integrität des Zustandes; Vertraulichkeit von dem Cloud-Provider 
        \begin{itemize}
            \item Vertraulichkeit: Die persönliche Ferienfotos wurden von Dritten erreicht und übermittelt.
            \item Integrität des Zustandes: da es keine andere Kopie der Fotos gibt, könnte man nicht sicher sein, dass die Fotos nicht verändert wurden.
            \item Die Vertraulichkeit des Cloud-Dienstes könnte auch verletzt wurden, wie z.B. ein Datenleck abhängig davon, wie die Fotos von dem Werber erhalten wurden. 
        \end{itemize}

        \item Nicht-Abstreitbarkeit(Integrität) wurde verletzt, wenn (ich) behaupten würde, dass ich die App nie gekauft habe.
    \end{enumerate}    

    \section{Strafbarkeit}

    \begin{enumerate}[(a)]
        \item H könnte mit 202a und 202d bestrafen werden. Er nutzt eine Schwachstelle aus und erreicht private Informationen und diese übermittelt. (Vertraulichkeit verletzt.)
        \item möglicherweise mit 202a strafbar, aber da alle Beteiligten das gewähren und auch der ganze Wettbewerb in einem privaten Netz stattfindet, sollte es nicht strafbar sein.
        \item 
        \begin{itemize}
            \item Der Student S mit 202b (und 202c) strafbar, da unberechtigter Zugang auf dem Netzwerk seines Nachbarn.
            \item Programmierer P mit 202c strafbar: Entwicklung von einem solchen Programm könnte als Vorbereitung angenommen werden.
        \end{itemize}
        \item
        \begin{itemize}
            \item 202a == unbefugter Zugang auf geschützter Servers
            \item 202d == Die Logindaten von Hintertüren im Darknet verkaufen, was dazu äquivalent wäre, Daten der Servers zu verkaufen.
            \item 303a == neue Software an Servers installieren, Serverkonfigurationen manipulileren. (Int. des Zustandes verletzt)
        \end{itemize}    
    \end{enumerate}    
\end{document}