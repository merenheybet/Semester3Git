\documentclass[11pt]{article}
\usepackage{graphicx} % Required for inserting images
\usepackage{indentfirst}
\usepackage[a4paper, margin=3cm]{geometry}
\usepackage[main=ngerman]{babel}
\usepackage{hyperref}
\usepackage[shortlabels]{enumitem}

\setlength{\parskip}{5pt}
\graphicspath{./images/}
\hypersetup{linktoc=all}

\title{BFS - Blatt 1}
\author{oz83upir (Jan Gremer) - lu08mika (Metin Eren Heybet)\\
 - om24esyc (Luka Jeremic) - ci24nony (Dennis Gehring)}
\date{03 November 2024}

\begin{document}

\maketitle

\section*{Aufgabe 6}

\subsection*{a) 7*a*b}

\begin{verbatim}
    // c(1)=a, c(2)=b, c(3)=Ergebnis
    
    1) C-LOAD 7
    2) STORE 4
    3) LOAD 1
    4) STORE 3

    // Erste for-loop: 7*a
    5) LOAD 3
    6) ADD 1
    7) STORE 3
    8) LOAD 4
    9) C-SUB 1
    10) STORE 4
    11) IF c(0) $>$  1 GOTO 5

    // Zweite for-loop: 7a * b
    12) LOAD 3
    13) STORE 5
    14) LOAD 2
    15) STORE 4
    16) LOAD 3
    17) ADD 5
    18) STORE 3
    19) LOAD 4
    20) C-SUB 1
    21) STORE 4
    22) IF c(0) $>$ 1 GOTO 16

    23) END
\end{verbatim}

Hinweis: Die for-loops haben die Bedingung "$>$ 1", da c(3):
Beim ersten Loop mit a, beim zweiten Loop mit 7a initialisiert wird.

Laufzeit = \boldmath$O(b)$\unboldmath. Die obere Loop für die Berechnung von 7a wird stets 7 mal ausgeführt.
 Die zweite Loop wird b mal ausgeführt. Deswegen hängt die Laufzeit von b ab.

\subsection*{b) c(1) zur Basis-7 umwandeln}

\begin{verbatim}
    // c(1)=a, c(2)=b, c(3)=Ergebnis
    
    1) C-LOAD 10
    2) STORE 4
    3) LOAD 1
    4) C-DIV 7
    5) STORE 2
    6) C-MULT 7
    7) STORE 3
    8) LOAD 1
    9) SUB 3
    10) IND-STORE 4
    11) LOAD 4
    12) C-ADD 1
    13) STORE 4
    14) LOAD 2
    15) STORE 1
    16) IF c(0) > 0 GOTO 4
\end{verbatim}

\section*{Aufgabe 7}
\subsection*{a) Berechnungen für die Eingaben: 0011 und 011}

\begin{itemize}
    \item $q_00011 \vdash Xq_1011 \vdash X0q_111 \vdash Xq_20Y1 
    \vdash q_4X0Y1 \vdash Xq_00Y1 \vdash XXq_1Y1 \vdash XXYq_11 \vdash XXq_2YY 
    \vdash Xq_2XYY \vdash XXq_3YY \vdash XXYq_3Y \vdash XXYYq_3 \vdash XXYYYq_5$
    \boldmath $\Longrightarrow$ \textbf{TM hält akzeptierend, $q_5$ ist ein gültiger Zustand, Eingabe wird akzeptiert.}

    \item \unboldmath $q_0011 \vdash Xq_111 \vdash Xq_3Y1 \vdash XYq_31$ \boldmath $\Longrightarrow$ \textbf{TM hält nicht akzeptierend, $q_3$ kein gültiger Zustand, Eingabe wird nicht akzeptiert}
\end{itemize}

\subsection*{b) akzeptierte Sprache}

$L = \{{0^n 1^n | n \geq 1}\} $

Begründung: Da die TM nur akzeptierend hält, wenn die Eingabe eine Anzahl n von Nullen gefolgt von n Einsen enthält (und nicht weiteres), wodurch das Band am Ende n X und n+1 Y enthält und dann die TM den gültigen Endzustand q5 erreicht.

Terminiert das Programm?\\
Das Programm hält in jedem Zustand für ein Blank, außer in q3. Dort wird allerdings in den gültigen Endzustand übergegangen.\\
Es muss also mit endlicher Eingabe ein Nicht-Halten erreicht werden. Dazu brauchen wir eine Endlosschleife in den Zustandsübergängen und dazu brauchen wir einen Zustandsübergang,
 der zu einem Zustand mit einer niedrigeren Nummer führt. Der einzige solche ist der Übergang von q4 auf q0. Somit ergibt sich die einzige mögliche Endlosschleife q0 $\rightarrow$ q1 
$\rightarrow$ q2 $\rightarrow$ q4 $\rightarrow$ q0. Hierbei wird allerdings im Zustand q0 eine 0 und im Zustand q1 eine 1 vernichtet, wodurch auch diese mögliche Endlosschleife keine Endlosschleife
ist und halten würde. Somit gibt es keine Endlosschleifen und die TM hält für jede Eingabe.
\end{document}